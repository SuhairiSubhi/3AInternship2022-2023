
\subsection{Model Driven Engineering}

\subsection{A Methodology Based on Model-Driven Engineering for IoT Application Development}
Model driven engineering eases the development of IoT Application. MDE allows model transformation in order to generate the code of the software applications for IoT. IoT systems must be physically or virtually interconnected and Service Oriented Architecture allows the deployment of such applications. SOA-based architecture is composed as 4 layers :
\begin{itemize}
    \item 
An Object Layer, which allows the identification, state management and data exchange of objects.
    \item
A Network Layer, which allows connection and communication between objects.
    \item A Service Layer, which allows management and creation of services used by users or softwares applications.
    \item
A Application Layer, which is responsible for delivering the applications to IoT users.

\end{itemize}

The paper introduces a methodology for the development of software applications for IoT that is composed of 4 phase : 
\begin{itemize}
    \item 
Analysis of business requirements
\item
Definition of the business logic
\item
Design of the integrated services solution
\item
Generation of the technological solution
\end{itemize}
This methodology reduces the time and costs in software development by implementing automatic and semi-automatic model transformations.

\subsection{Design of a domain specific language and IDE for Internet of things applications}
The paper offers a solution to the complexity of designing IoT applications with a wide range of wireless sensory networks (WSNs), devices, communication media, protocols and operating systems. In that regard, DSL-4-IoT Editor-Designer has been developed. The solution use : 
\begin{itemize}
    \item 
PervML,  a DSL that allow developers to describe pervasive system in a technology independent way 
\item
DiaSuite, a tool suite providing DSL designing tools, Java code generation, 2D renderer and deployment framework. 
\item
Node-RED, a visual tool allowing complex wirings and connections for IoT applications and objects.
\item
OpenHAB, a general-purpose framework for smart home.
\end{itemize}
And allows its users to build DSLs with a high level visual programming language built in JS.

\subsection{MDE4IoT: Supporting the Internet of Things with Model-Driven Engineering}
IoT applications brings many challenges for developers, among them , supporting complexity and heterogeneity management, supporting collaborative development, maximizing reusability of design artifacts, and providing self-adaptation of IoT systems. MDE4Iot is a framework allowing  high-level abstraction and separation of concerns to manage heterogeneity and complexity of Things, enabling collaborative development, enforcing reusability of design artifacts, and automation, in terms of model manipulations, enabling runtime self-adaptation. 

\subsection{Model Driven Development for Internet of Things Application Prototyping}
This paper discusses the use of MDE in the development of IoT Applications, and provides a different viewpoint on the conception of such applications by placing domain modeling and object virtualization in the center of the architecture.

\subsection{Domain Specification Language}
TBC

\subsection{Artificial Intelligence} 
TBC

\subsection{IoT Definition : }
Do not forget that we are particularly working on IoT domain application and specifically on signal synchronisation for healthcare use case (EEG and ECG).
