
\subsection{Model Driven Engineering}
TBC

\subsection{Domain Specification Language}
TBC

\subsection{Artificial Intelligence} 
TBC

\subsection{IoT Definition : }
%Do not forget that we are particularly working on IoT domain application and specifically on signal synchronisation for healthcare use case (EEG and ECG).
  \begin{itemize}
  
      \item IOT stands for Internet Of Things, it can be defined as " a self configuring and adaptive system consisting of networks of sensors and smart objects whose purpose is to interconnect ‘all’ things" according to the IEEE Iot Community. In the article \uline{Building Caring Healthcare Systems in the Internet of Things}, three general classes for use cases of IOT in healthcare are defined. In this domain, Iot is mainly used to collect data for these purposes : Tracking humans, tracking things and tracking both.

      \item One important aspect of Iot, especially for healthcare Iots, is clock synchronisation. In the article \uline{A System for Clock Synchronization in an Internet of Things}, a solution is proposed for this challenge. This solution is SPoT, a packet exchange protocol, and it has advantages in comparison to other standard protocols.

      \item IOT applications\newline
      \begin{itemize}
        \item The article \uline{Building Caring Healthcare Systems
        in the Internet of Things} shows a way to describe, specify and implement an IOT application. It underlines the need for an expertise in healthcare domain and         a strong collaboration between engineers and healthcare workers.
        \item One exemple of an Iot implementation is shown in the article : \uline{Implementation of a portable device for real-time ECG signal analysis}. In this             article, this Iot application for real-time
        electrocardiogram (ECG) acquisition is described.
        \item Finally, the article \uline{Create Your Own Internet of Things} targets anyone who wants to make their own iot device by proposing a selection of           hardwares, platforms, and programming langages for Iot creation. It compares different aspects such as prices, connectivity or compatibility.
      \end{itemize}
  \end{itemize}
