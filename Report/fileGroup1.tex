In this part we will discuss the question : how to find the best model using artificial intelligence? This involves a combination of approaches, including model-based machine learning (ML), incremental learning methods, and the integration of machine learning into domain modeling. 
Model-based machine learning involves specifying ML problems using a dedicated modeling language and generating the corresponding ML code automatically, allowing for the creation of highly tailored models for specific scenarios and rapid prototyping and comparison of alternative models\cite{evolutionMDE}. In \cite{mdApproach} a new approach is presented. It is based on the domain-specific modeling (DSM) methodology and involves the creation of a domain-specific modeling language (DSML) that can be used to generate ML models as well as code for their implementation.
Incremental learning methods, such as those based on Hoeffding bounds and Hoeffding trees, have been proposed as a way to efficiently process and analyze massive data streams in real-time\cite{evolutionMDE}. 
The integration of machine learning into domain modeling involves decomposing ML into small, reusable units called microlearning units \cite{evolutionMDE}, which can be modeled and used alongside domain data, allowing for the flexible combination of learned behaviors and domain knowledge. An example of machine learning integration is given in the paper ThingML \cite{ThingML}.  ThingML is an open source MDE solution for CPS and IoT systems that uses a domain-specific modeling language, methodology, and tool to allow for the specification of distributed systems using components, composite state machines, and an action language. The paper proposes the integration of Machine Learning (ML) into the ThingML approach in order to enable the development of data-driven behavior in Cyber-Physical System (CPS) and IoT systems.
the availability of large and high-quality datasets, such as the ModelSet dataset \cite{modelset}, is crucial for the application of machine learning in model-driven engineering.


\subsection{The use of AI in MDE}
Machine learning (ML) is a widely used approach for enabling computers to learn from and make decisions based on data. There are many tools and frameworks available for implementing ML algorithms, such as TensorFlow, GraphLab, and Infer.NET. These tools allow for the expression of ML algorithms at a higher level of abstraction, and often include an execution engine for running the algorithms on a variety of devices.
There is also a growing trend towards using a model-based approach for ML, in which ML problems are specified using a dedicated modeling language and the corresponding ML code is generated automatically. This allows for the creation of highly tailored models for specific scenarios and rapid prototyping and comparison of alternative models.
Incremental learning methods, such as those based on Hoeffding bounds and Hoeffding trees \cite{evolutionMDE}, have been proposed as a way to efficiently process and analyze massive data streams in real-time. Tools like MOA provide implementation and support for these methods.
Other research has focused on weaving ML into domain modeling, allowing ML algorithms to be seamlessly integrated into the process of creating models for a particular domain or application. This approach involves decomposing ML into small, reusable units called microlearning units, which can be modeled and used alongside domain data. This approach allows for the flexible combination of learned behaviors and domain knowledge, and can be more accurate and efficient than learning global behaviors.In this paper, the authors propose a method for seamlessly integrating machine learning into domain modeling by decomposing machine learning into microlearning units that are modeled together with and at the same level as the domain data. The approach is demonstrated using a smart grid case study, showing that it can be significantly more accurate than learning a global behavior, while still being fast enough for live learning.


The application of machine learning (ML) to software engineering has gained significant attention in recent years, with a focus on three types of ML models: code-generating models, representational models of code, and pattern mining models. These models have various applications, such as code recommendation systems, inferring coding conventions, clone detection, and code-to-text and text-to-code translation.There are several existing datasets related to software models, including the LindholmenDataset and ModelSet, as well as datasets of OCL expressions, BPMN models, and APIs classified using Maven Central tags. However, many of these datasets have limitations, such as a variety of formats and versions, invalid or poor quality models, or a lack of labels.
There have been several studies on the application of AI and ML to address MDE problems, including search-based algorithms and reinforcement learning for co-evolution and model repair, respectively. There have also been efforts to classify UML class diagrams and Ecore meta-models, and to use clustering techniques on collections of models. However, many of these studies are not easily replicable due to the lack of available datasets or the difficulty in processing the models.
ModelSet dataset, which includes over 10,000 labeled models, aims to address the need for a large and high-quality dataset of software models for use in ML research. The dataset is composed of models in the Ecore metamodeling language, and includes labels for the model category and domain, as well as structural and textual features. \cite{modelset} demonstrate the use of the dataset in a classification task, showing that it can be used to improve the accuracy of existing ML models for classifying Ecore models.


\cite{mdApproach} propose a novel approach to integrating machine learning (ML) models with software and systems engineering models, particularly in the context of model-driven software engineering (MDSE). The approach is based on the domain-specific modeling (DSM) methodology and involves the creation of a domain-specific modeling language (DSML) that can be used to generate ML models as well as code for their implementation. The DSML is demonstrated through a case study in the Internet of Things (IoT) and cyber-physical systems domains, but is not limited to these specific verticals.
There are several directions for future work, including the support of semi-supervised ML, the extension of the supported ML methods to include kernel methods and advanced artificial neural network architectures, the addition of more target platforms and programming languages, and the inclusion of advanced autoML functionalities.

Domain analysis \cite{mdApproachMonitoring}  is about studying and understanding a particular problem or domain in order to develop a solution or model that can effectively solve the problem. In the context of machine learning, domain analysis involves understanding the properties of the data, the target variables, and the relationships between the observed and target variables, in order to develop a model using techniques such as supervised learning. However, in non-stationary environments, the joint probability distribution between the observed and target variables may change over time, leading to differences between the training and test sets, which may affect the performance of the model. To address this problem, researchers have developed methods that adapt to non-stationary environments, such as online learning algorithms or hybrid approaches.

The domain meta-model refers to the larger system that is required to effectively apply machine learning techniques in a commercial setting, considering factors such as latency, cost, compatibility, and scalability. The domain meta-model also needs to consider the process of continuous learning, which involves regularly updating the model with new data in order to maintain its performance. 

\subsection{The use of AI in IoT}
The Internet of Things (IoT) is a network of interconnected devices that can communicate with each other and with external systems. Cyber-Physical Systems (CPS) are systems of systems that combine both physical and virtual elements. Artificial Intelligence (AI) is increasingly being used in the development of IoT and CPS systems to enable them to have cognitive capabilities such as learning. Model-Driven Engineering (MDE) is a promising approach that provides both abstraction and automation for the specification, design, development, analysis, verification and maintenance of CPS and IoT systems. ThingML \cite{ThingML} is an open source MDE solution for CPS and IoT systems that uses a domain-specific modeling language, methodology and tool to enable the specification of distributed systems using components, composite state machines and an action language. 

In the field of IoT, the increasing number of connected devices has led to the need for efficient network management schemes. Hierarchical network architectures such as H-CRANs and cooperative cloud-edge computing have been developed to provide efficient network-wide management and address the computing demands of IoT devices. Edge computing, in particular, has emerged as a way to process tasks at the network edge near mobile users in order to reduce latency and improve service quality for delay-sensitive applications. To minimize delay in mobile computing networks, various approaches have been proposed, such as task offloading, content caching, and reducing data exchange among IoT systems. Machine learning can also be used to improve computing performance by adapting service provisioning based on past events. In \cite{DelaySensitiveIoT} a decision tree model was proposed to classify tasks as delay-sensitive or delay-insensitive, and a priority queuing model was introduced to prioritize delay-sensitive tasks in order to avoid long queuing delays. The decision tree model was trained and tested using simulated data sets, and the accuracy of the model was investigated. The proposed model was shown to effectively classify tasks and reduce queuing delay in the edge device.
